% !TEX root = Main.tex

%--------------------------------------------------------------------------------
%	PACKAGES
%--------------------------------------------------------------------------------

\usepackage{amsmath,amsfonts,amssymb,amscd,amsbsy,amsthm}
\usepackage{mathtools,textcomp,stackrel}
\usepackage{xfrac}
\usepackage{bm}
\usepackage{esint}
\usepackage{enumerate}

\DeclareMathAlphabet{\mathpzc}{OT1}{pzc}{m}{it}
\DeclareSymbolFont{cmletters}{OT1}{cmr}{m}{n}
\DeclareMathSymbol{\Upsilon}{\mathalpha}{cmletters}{"7}

\usepackage[bbgreekl]{mathbbol}
\DeclareSymbolFontAlphabet{\mathbb}{AMSb}
\DeclareSymbolFontAlphabet{\mathbbl}{bbold}

\usepackage[mathscr]{euscript}

%--------------------------------------------------------------------------------
%	WRAPPERS
%--------------------------------------------------------------------------------

\newcommand{\fractd}[2][]{\frac{\dd #1}{\dd #2}}
\newcommand{\fracpd}[2][]{\frac{\partial #1}{\partial #2}}
\newcommand{\uu}[3][]{\mkern1mu{}_{#1} #2_{#3}}

%--------------------------------------------------------------------------------
%	MAPS
%--------------------------------------------------------------------------------

\let\sto\to
\renewcommand{\to}{\xrightarrow{\;\hphantom{\cong}\;}}
\renewcommand{\mapsto}{\xmapsto{\;\hphantom{\cong}\;}}
\newcommand{\homeoto}{\xrightarrow{\;\cong\;}}

%--------------------------------------------------------------------------------
%	SYMBOLS
%--------------------------------------------------------------------------------

% USUAL SETS
\newcommand{\calT}{\mathcal{T}}
\newcommand{\calU}{\mathcal{U}}
\newcommand{\calV}{\mathcal{V}}
\newcommand{\calC}{\mathcal{C}}
\newcommand{\calA}{\mathcal{A}}

% FIELDS OF NUMBERS
\newcommand{\bbN}{\mathbb N}
\newcommand{\bbZ}{\mathbb Z}
\newcommand{\bbQ}{\mathbb Q}
\newcommand{\bbR}{\mathbb R}
\newcommand{\bbC}{\mathbb C}
\newcommand{\bbF}{\mathbb F}

\newcommand{\bblN}{\mathbbl N}
\newcommand{\bblZ}{\mathbbl Z}
\newcommand{\bblQ}{\mathbbl Q}
\newcommand{\bblR}{\mathbbl R}
\newcommand{\bblC}{\mathbbl C}
\newcommand{\bblF}{\mathbbl F}


%--------------------------------------------------------------------------------
%	MATH OPERATORS
%--------------------------------------------------------------------------------

\DeclareMathOperator{\sgn}{sgn}
\DeclareMathOperator{\diag}{diag}
\DeclareMathOperator{\dd}{d\mkern-1mu}
\DeclareMathOperator{\cc}{c.c.}
\DeclareMathOperator{\im}{im}

%--------------------------------------------------------------------------------
%	GRAPHS
%--------------------------------------------------------------------------------

% usage:
% \shortexact[f][g]{A}{B}{C},
%
%			 f    g
% for 0 -> A -> B -> C -> 0,
\DeclareDocumentCommand{\shortexact}{O{} O{} mmmm}{
\xymatrix{
	0\ar[r] & #3\ar[r]^-{#1} & #4\ar[r]^-{#2} & #5\ar[r] & 0#6
}}
% exactly the same, but for 0 -> A -> B -> C
\DeclareDocumentCommand{\leftexact}{O{} O{} mmmm}{
\xymatrix{
	0\ar[r] & #3\ar[r]^-{#1} & #4\ar[r]^-{#2} & #5 #6
}}
% ... and the same, for A -> B -> C -> 0
\DeclareDocumentCommand{\rightexact}{O{} O{} mmmm}{
\xymatrix{
	#3\ar[r]^-{#1} & #4\ar[r]^-{#2} & #5\ar[r] & 0#6
}}


%--------------------------------------------------------------------------------
%	DELIMITERS
%--------------------------------------------------------------------------------

\DeclarePairedDelimiter\angl{\langle}{\rangle}
\DeclarePairedDelimiter\abs{\lvert}{\rvert}
\DeclarePairedDelimiter\norm{\lVert}{\rVert}
\DeclarePairedDelimiter\bkt{[}{]}
%\DeclarePairedDelimiter\set{\lbrace\,}{\,\rbrace}
\DeclarePairedDelimiterX\set[1]{\lbrace}{\rbrace}{\mkern1mu#1\mkern1mu}
\DeclarePairedDelimiterX\setst[2]{\lbrace}{\rbrace}{\mkern1mu#1\mkern3mu\delimsize\vert\mkern3mu#2\mkern1mu}

\DeclarePairedDelimiter\bra{\langle}{\rvert}
\DeclarePairedDelimiter\ket{\lvert}{\rangle}
\DeclarePairedDelimiterX\braket[2]{\langle}{\rangle}{#1 \delimsize\vert #2}
\DeclarePairedDelimiterX\braOket[3]{\langle}{\rangle}{#1 \delimsize\vert #2 \delimsize\vert #3}

\renewcommand{\vec}[1]{\mathbf{#1}}

%--------------------------------------------------------------------------------
%	THEOREMS
%--------------------------------------------------------------------------------

%\catcode`\"=13
%\newcommand{"}[1]{^{(#1)}}
\theoremstyle{definition}
\newtheorem{theorem}{Theorem}
\newtheorem*{theorem*}{Theorem}
\newtheorem{lemma}{Lemma}
\newtheorem*{lemma*}{Lemma}
\newtheorem{corollary}{Corollary}
\newtheorem{proposition}{Proposition}
\newtheorem{exercise}{Exercise}
\newtheorem{example}{Example}
\newtheorem{definition}{Definition}
\newtheorem*{notation}{Notation}
\newtheorem*{claim}{Claim}
\theoremstyle{remark}
\newtheorem*{notabene}{Note}
\newtheorem*{remark}{Remark}
\newtheorem*{fact}{Fact}

\numberwithin{theorem}{section}
\numberwithin{lemma}{section}
\numberwithin{corollary}{section}
\numberwithin{proposition}{section}
\numberwithin{exercise}{section}
\numberwithin{example}{section}
\numberwithin{definition}{section}

